\documentclass[letterpaper,12pt]{article}
\usepackage{amsthm,amssymb,amsmath}
\usepackage[top=25mm, bottom=25mm, left=15mm, right=15mm]{geometry}
\usepackage{graphicx}
\usepackage{subfigure}
\usepackage{array}

\usepackage[pagebackref=true,colorlinks,linkcolor=blue,citecolor=magenta]{hyperref}
\usepackage{fancyhdr}
%\settextfont[Scale=1.0]{XB Roya}
%\setlatintextfont[Scale=1.0]{XB Roya}
%\setlatintextfont[ExternalLocation,BoldFont={lmroman10-bold},BoldItalicFont={lmroman10-bolditalic},ItalicFont={lmroman10-italic}]{lmroman10-regular}
%\defpersianfont\titr[Scale=1.1]{XB Titre}
%\defpersianfont\nastaliq[Scale=1.1]{IranNastaliq}
%\defpersianfont\traffic[Scale=1]{B Traffic}
%\defpersianfont\roya[Scale=1]{XB Roya}

\def \CHWName {CPSC 500, Homework 5, Fall 2013}
\def \myName {Alireza Shafaei (78428133)}

\fancyhead{}
\fancyfoot{}
%\fancyhead[RO,RE]{\roya{\myName}\vspace{0.08cm}}
\fancyhead[RO,RE]{\textbf{\myName}\vspace{0.00cm}}
\fancyhead[LO,LE]{\textbf{\CHWName}\vspace{0.00cm}}
\fancyfoot[CO, CE] {\thepage}
%\fancyfoot[LO, LE] {\small{\today}}
\renewcommand{\headrulewidth}{0.3pt}
%\renewcommand{\footrulewidth}{0.3pt}
\newcommand{\threepartdef}[6]
{
	\left\{
		\begin{array}{lll}
			#1 & \mbox{ } #2 \\
			#3 & \mbox{ } #4 \\
			#5 & \mbox{ } #6
		\end{array}
	\right.
}

\newcommand{\fourpartdef}[8]
{
	\left\{
		\begin{array}{lll}
			#1 & \mbox{ } #2 \\
			#3 & \mbox{ } #4 \\
			#5 & \mbox{ } #6 \\
			#7 & \mbox{ } #8
		\end{array}
	\right.
}


\newcommand{\twopartdef}[4]
{
	\left\{
		\begin{array}{ll}
			#1 & \mbox{if } #2 \\
			#3 & \mbox{if } #4
		\end{array}
	\right.
}


\begin{document}
\pagestyle{fancy}

In order to prepare this submission, the following resource(s) has been used:
\begin{itemize}
	\item Discussion with fellow students: \textit{Victor Gan}, \textit{Sarah Huber}, and \textit{Evan Peng}.
\end{itemize}

\begin{enumerate}
\item (from DPV Problem 7.28) A linear program for shortest path. Suppose we want to compute the shortest path from node $s$ to node $t$ in a directed graph with edge lengths $l_e > 0$.

\begin{enumerate}
	\item Show that this is equivalent to finding a s-t flow $f$ that minimizes $\sum l_ef_e$ subject to $size(f) = 1$. There are no capacity constraints.\\
	\textbf{Lemma 1.} If there exists a s-t flow $f$ that minimizes $\sum l_ef_e$ subject to $size(f) = 1$, then there exists an equivalent flow $f'$ which consists of only one path from $s$ to $t$.\\
	\textbf{Lemma 1 Proof.} If the flow $f$ is already a single path, we are done. Otherwise, Let's select a shortest path $p_s$, and then pick a random path $p_r$ from $s$ to $t$ in our flow. We select $Q_p$ to be the minimum $f_{e\in{}p_r}$. Given the linearity of the objective function, $Q_p$ could be considered as a flow which has gone from $s$ to $t$ with partial penalty $\sum_{e\in{}p_r} l_eQ_p = Q_p \sum_{e\in{}p_r}l_e$. $\sum_{e\in{}p_r}l_e$ can't be greater than $\sum_{e\in{}p_s}l_e$, which is the length for the shortest path, otherwise we can direct the flow $Q_p$ to $p_s$ and lower the overall objective which will contradict the first assumption of having a flow satisfying the mentioned conditions. It also can't be smaller because that will contradict the assumption that $p_s$ is a shortest path. Thus we can direct the flow $Q_p$ from $p_r$ to $p_s$ without losing anything. Since $Q_p$ was the smallest flow in path $P_r$, changing the flow path will eliminate at least one edge from the $f$. Since the number of edges are finite, after repeating this process we will end up with a single path (and flow $f'$) from $s$ to $t$ which still satisfies the conditions and is not worse than the flow $f$.\\
	\textbf{Problem Proof.} First I prove that finding a $s$-$t$ flow $f$ satisfying the above conditions implies finding a shortest path from $s$ to $t$. Given the Lemma 1, any $f$ satisfying these constraints will have an equivalent flow $f'$ which is only a single path $p$ from $s$ to $t$. Thus the flow along that path will be 1, because the flow size is 1. Thus the objective function will become $min (\sum_{e\in{}p}l_e)$, which in conjunction with $p$ being a path from $s$ to $t$, is the definition for the shortest path from $s$ to $t$. For the other way, let's assume the shortest path is $p$. If $p$ is the flow satisfying the constraints, we're done. Otherwise, let's assume the flow $f$ is the best answer to the problem. Considering the Lemma 1, there exists a single path $p'$ which also satisfies this condition and forms the flow $f'$ which is equivalent to flow $f$. Given the assumptions $length(p')$ must be less than or equal to $length(p)$, because $p$ is the shortest path. It also can't be longer than $p$, because then $p$ will yield a better flow $f''$ which contradicts the initial assumption. Thus the shortest path must be a single path answer to the above optimization problem. So these two problems are equivalent.
	
	\item Write the shortest path problem as a linear program.\\
	\textbf{Answer.} We simply have to convert the above flow problem into a linear program.\\
	\begin{gather}
		\text{Minimize } Z = \sum l_ef_{uv}  \text{ subject to:}\\
		\sum_{v \in V} f_{sv} = \sum_{v \in V} f_{vt} = 1 \text{ (Define s and t property)}\\
		\text{For each $u \in{} V$ :} \sum_{v \in V} f_{uv} = \sum_{v \in V} f_{vu} \text{ (Flow conservation)}\\
		f_{uv} \geq 0
	\end{gather}
	\item Show that the dual LP can be written as:
	\begin{gather*}
		\text{max }x_s - x_t\\
		x_u - x_v \leq l_{uv}\text{ for all }(u, v) \in E
	\end{gather*}
	\textbf{Answer.}Expanding the linear program we'd have:
	\begin{gather}
		\text{Minimize } Z = \sum l_ef_{uv}  \text{ subject to:}\\
		\sum_{v \in V} f_{sv} = 1 \label{ls} \\
		\sum_{v \in V} f_{vt} = 1 \label{lt} \\
		\text{For each $u \in{} V$ :} \sum_{v \in V} f_{uv} - \sum_{v \in V} f_{vu} = 0  \label{lu}
	\end{gather}
	By multiplying \ref{ls} by $X_s$ and \ref{lt} by $-X_t$ and \ref{lu} by $X_u$ and then summing all the equalities, we'd get:
	\begin{gather}
		\text{Minimize } Z = \sum l_ef_{uv}  \text{ subject to:}\\
		X_s \sum_{v \in V} f_{sv} = X_s\\
		-X_t \sum_{v \in V} f_{vt} = -X_t\\
		\text{For each $u \in{} V$ :} X_u \sum_{v \in V} f_{uv} - X_u \sum_{v \in V} f_{vu} = 0\\
		\Rightarrow \sum_{(u, v)\in E} f_{uv}(X_u - X_v) = X_s - X_t \label{fsum}
	\end{gather}
	\ref{fsum} results from the fact that for each edge $(u, v)$ we have exactly $-X_v$ and $+X_u$ in the other equalities, because each edge appears exactly once in the entrance of vertex $v$ ($-X_v$), and on the departure from one vertex $u$ ($+X_u$).\\
	As long as $X_u - X_v \leq l_{uv}$, the function $\sum_{(u, v)\in E} f_{uv}(X_u - X_v)$ will be less than or equal to the objective function $\sum_{(u, v)\in E} f_{uv}l_{uv}$ (Note that $f_{uv}\geq{}0$). So this function (subject to constraint $X_u - X_v \leq l_{uv}$) bounds the minimum value for the objective function $\sum_{(u, v)\in E} f_{uv}l_{uv}$. So by maximizing the lower bound for the objective function, we'd be able to identify the minimum value for the objective function. Luckily, due to the \ref{fsum}, this bounding function is also equal to $X_s - X_t$, thus it suffices to maximize $X_s - X_t$ subject to the constraint $X_u - X_v \leq l_{uv}$. So the dual LP can be formulated as:
	\begin{gather*}
		\text{max }X_s - X_t\\
		X_u - X_v \leq l_{uv}\text{ for all }(u, v) \in E
	\end{gather*}
\end{enumerate}
\item Let $G$ be a bipartite graph. We want to find a minimum edge cover; show
how to solve this problem using a maximum flow algorithm.\\
\textbf{Answer.}
First, we add a directed edge from $S$ to vertices in partition 1. For each vertex $v$ that we add an edge to, we set the flow capacity to be equal to $c(S, v) = |n(v)|-1$, where $n(v)$ is the set of neighbors of $v$. Then, from each vertex $u$ in partition 2, we add a directed edge to $T$ with flow capacity of $c(u, T) = |n(u)| - 1$. Finally, we set the capacity of edges between partition 1 and 2 to be $1$.\\

\begin{figure}[ht!]
	\centering
	\includegraphics[width=10cm]{untitled2.png}
	\caption{Converting the edge-cover bipartite graph problem to a network flow problem.}
\end{figure}

Now I claim the maximum flow in this graph yields the minimum edge-cover. The edges that are not actually part of the flow, are the edges we need for the covering problem.
\textbf{Proof.}
\begin{enumerate}
	\item \label{pa} For each possible flow in this graph, there's an equivalent edge cover solution.\\
	\textbf{Proof.} Since each arc going into a vertex $v$, has the capacity of $|n(v)|-1$, in any flow there always exists an edge from $v$ to partition 2 that's not used (otherwise it'll be breaking the capacity constraint of the entering edge). The same argument goes for the second partition; in any flow from $u$ there exists at least an edge coming from partition 1 which is not used. Therefore, all the unused edges in this flow will be creating an edge cover.
	\item \label{pb} For any edge cover, there exists at least an equivalent flow.\\
	\textbf{Proof.} Simply remove the edges of the cover from the original graph and then solve the max flow problem.
	\item \label{pc} The equivalent edge cover of any flow from $S$ to $T$ has the size of
	\begin{equation}
		|E| - size(f)
		\label{formula}
	\end{equation}
	where $|E|$ is the number of edges.\\
	\textbf{Proof.} Given that
	\begin{gather}
		size(f) = \sum_{v_i \in partition 1} f(S, v_i)\\
		\forall_{v_i \in partition 1} f(S, v_i) \leq c(S, v_i) < |n(v_i)|
	\end{gather}
	The number of edges that are not involved in the flow will be the total number of edges minus flow size.
	\item \textbf{Conclusion.} In order to minimize the number of edges for the edge cover problem, we have to minimize the \ref{formula}, which is equivalent to maximizing the $size(f)$ which leads us to the max flow problem. Thus solving the max-flow problem will give us the minimum edge cover solution.
\end{enumerate}

\item  Each of two players hides a nickel (5 cents) or a dime (10 cents). If the two coins match then A gets both; if they don't match then B gets both. 
\begin {enumerate}
	\item Show the payoff matrix. (The payoff is the additional money a player gets.)\\
	\textbf{Answer.} For A:
	\begin{table}[ht!]
		\centering
		\begin{tabular}{|c|c|c|}
		\hline
		 	&	5	&	10\\ \hline
		 5	&	+5	&	-5\\ \hline
		 10 &	-10 &	+10\\ \hline
		\end{tabular}
	\end{table}
	\item  Describe the optimal strategy for A using a linear program. What is the optimal strategy?
	\textbf{Answer.} Let's say for A: $P(X=5)=x$ and $P(X=10)=y$. For B, the options will be $5x-10y$ and $-5x+10y$. Obviously, B will be trying to minimize the loss; and A must maximize the minimum loss for A to guarantee the gain. The following constraints will show us the solution.
	\begin{gather}
		\text{Maximize }P=z+0x+0y\text{ subject to}\\
		z-5x+10y <= 0\\
		z+5x-10y <= 0\\
		x + y = 1\\
		0 <= x, y <= 1\\
		\Rightarrow x = \frac{2}{3}, y = \frac{1}{3}, p = 0
	\end{gather}
	So player A must pick 5 with probability 2/3, and 10 with probability 1/3; and doing so, he can guarantee to not lose anything.
	\item What is the optimal strategy if instead of a nickel or a dime the players hide a c-cent coin or a d-cent coin?\\
	\textbf{Answer.}
	\begin{gather}
		\text{Maximize }P=z+0x+0y\text{ subject to}\\
		z-cx+dy <= 0 \label{p1}\\
		z+cx-dy <= 0 \label{p2}\\
		x + y = 1 \label{p3}\\
		0 <= x, y <= 1\\
		\ref{p1}, \ref{p2} \Rightarrow z <= 0\\
		z=0 \Rightarrow z=0=cx-dy \Rightarrow \frac{c}{d}x=y \label{p4}\\
		\ref{p3}, \ref{p4} => \frac{c}{d}x=1-x \Rightarrow x=\frac{d}{c+d}, y=\frac{c}{c+d}.
	\end{gather}
	So we must pick c with probability $\frac{d}{c+d}$, and d with probability $\frac{c}{c+d}$; and doing so will guarantee no loss.
\end{enumerate}

\end{enumerate}

\end{document}
